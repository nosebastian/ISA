\documentclass[11pt, a4paper]{article}
\usepackage{amsmath, amsthm, amssymb}
\usepackage{times}
\usepackage[utf8]{inputenc}
\usepackage[left=2cm, top=3cm, text={17cm, 24cm}]{geometry}
\usepackage{alltt}
\usepackage[hidelinks]{hyperref}
\usepackage[czech, linesnumbered]{algorithm2e}

\begin{document}
\begin{titlepage}
\begin{center}

{\Huge\textsc{FACULTY OF INFORMATION TECHNOLOGY\\BRNO UNIVERSITY OF TECHNOLOGY}}
\vspace{\stretch{0.382}}

{\LARGE  Network Applications and Network Administration (ISA)}{\Huge\\DNS resolver}

\vspace{\stretch{0.618}}
\end{center}
{\Large \today \hfill Marek Vaško (xvasko16)}
\end{titlepage}

\tableofcontents
\newpage

\section{Introduction}
The aim of this project was to create network aplication which is able to query DNS queries. 
Application should support domain resolve in Ip v4 and ip v6 and reverse queries. It should 
be able to handle A, AAAA and CNAME records. Queries are sent to server selected by user and
server answers are then returned and formated to standart output. Client needs to support 
comunication over UDP.
\section{Protocol description}
DNS protocol as described in RFC1035 is aimed to provide a mechanism for naming services. For user
DNS is a protocol which retrieves information associated with particular domain name from a 
database on a nameserver. Comunication between client and server goes on a port number 53. Messages
can be sent to server using UDP or TCP datagrams. UDP datagrams are limited in size to 512 bytes if  
data is of a larger size, flag signalizing data truncation is set. UDP datagrams are the main form
of comunication with nameserver, but since they can be lost during their route thus are unreliable, 
retransmission policy should be apppliad in case of no response. Comunication over TCP protocol 
although possible is not considered in this scenario.
\subsection{Datagram description}
Single datagram can be divided into several parts which can then be implemented into their separate 
modules. Parts include header, questions section, answers section, authorities section and additionals
section. These parts are serialized in the same order as described. Different parts of datagram consist
of other subparts which are questions for section questions and resource records for sections answers, 
authorities and additionals.
\subsubsection{Header}
Header section coitains information about dns query identification number, whenever message is response or query, opcode,
whenever message is authoritative answer (responding nameserver is authority), whenever message is truncated,
whenever client wishes to use recursion (recursion desired), whenever recursive queries are available on server,
response code (information about what error occured if it occured zero if not), questions count, answers count,
authoritative answers count and additional answers count.
\subsubsection{Question}
Question is format presented in questions section, single question section may contain multiple questions. Content of
question siglanazes to server what information is client asking for and for what hostname. First question field is the
what hostname part, this part is domain name represented by sequence of labels format of which is described in a later 
section of this document. Domain name is being used for standard forward lookup whel its content is domain name of server. 
It is also used for reverse DNS lookup via PTR record when in case of IPv4 it contains addres represented by sequence 
of 4 decimal numbers form 0 to 255, representing reverse representation of ip addres, followed by .in-addr.arpa. In case of
reverse lookup of IPv6 addres content of domain name is series of hexadecimal values from 0 to F, each representing 4 bites of
addres in reverse order separated by dots followed by .ip6.arpa. Other section of question contain question type of questio
(what record type is client asking for form server) and question class (information describing what class of records should server 
search for ie. Internet class).
\subsubsection{Resource record}
Resource record is used to represent responses of DNS server. First field describes what domain name does this record belong to.
And is in same format as domain names in DNS question. Other fields include type of resource record which is similiar to type of
question but exclude options like all records (more info can be found in RFC1035), class of resource record once again similiar to
class of question but excludes some options, time to live which signifies time of record expiration in seconds, length of resource 
data (rdata) and resource data itself. Resource data has specific form for every type of record. Resource data structures interesting
for our implementation are CNAME, A and AAAA. CNAME contains only domain name and its length is variable. A record contains IPv4 address
in binary form with length of 4 bytes. AAAA resord contains IPv6 address in binary form with lenght of 16 bytes.
\subsubsection{Domain name}
Domain name representation is used by multiple parts of DNS protocol, thus it can be considered its own structure. Domain names in DNS
are represented by sequence of labels. Each label begins with one byte which tells length of this label, next follow specified number of 
characters of specified count. The end of domain name is represented by zero byte. Labels represent different parts of domain name (subdomains),
each part separated by dot is considered label. Order of labels is the same as for subdomains order in domain name. Pointer is special case of 
domain name representation. Pointer is signalized by first 2 bites being ones (eg. 0xc0 is pointer), value of poiter is then stored in next 6 bits 
of first byte and the value of second byte. Poiter is refference position from start of a datagram.
\section{Application architecture}

\section{Application compilation}
\section{Application usage}

\section{Comertialy available resolvers}



\end{document}